\muzikajxo[
titolo-he     = {השקדיה פורחת},
titolo-eo     = {La migdalarbo floras},
titolo-xx     = {},
ikono         = {🌸},
komponisto-he = {מנשה רבינא},
komponisto-eo = {Menaŝe Ravina},
komponisto-xx = {},
noto-he       = {השם „שְׁקֵדִיָּה” לעץ השקד מקורו בשירי הילדים העבריים של תחילת המאה העשרים. המילה הופיעה בשני שירי ילדים מוקדמים: האחד הוא „לשנה טובה, שקדיה” מאת לוין קיפניס משנת 1920; השני, והמוכר יותר, הוא „השקדיה פורחת” מאת ישראל דושמן. ככל הנראה הראשון הוא שחידש את המילה, אבל השני, שנהיה מוכר ואהוב יותר, הוא כנראה האחראי להפצת המילה.\pnurl{zemer.co.il/song.asp?id=244}\,\pnurl{safa-ivrit.org/flora/shkediya.php}},
noto-eo       = {Noto pri la hebrea nomo de la kanto, «Ha Ŝkedija Poraĥat»: la hebrea nomo «ŝkedija» por la migdalarbo («(ec) ŝaked») originas el la hebreaj infanaj kantoj de la frua 20-a jarcento, unu el ili estas tio ĉi.},
]{ŝkedija}

\begin{tabulaturo}
	\takto{}{\c}{\x}\par
	\takto{}{\c\c\c\e\g\z\e\e}{\x}\par
	\takto{}{\g\g\a\a\g\z\e\z}{\x}\par
	\takto{}{\c\d\e\c\f\a\g\z}{\x}\par
	\takto{}{\g\g\a\g\e\g\c\z}{\x}\par
	\takto{}{\C\C\a\a\g\z\e\z}{\x}\par
	\takto{}{\f\f\e\d\g\z\z\z}{\x}\par
	\takto{}{\C\C\a\a\g\z\e\z}{\x}\par
	\takto{}{\f\f\e\d\c\z\z\z}{\X}
\end{tabulaturo}
