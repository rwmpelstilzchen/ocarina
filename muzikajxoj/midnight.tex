\muzikajxo[
titolo-he     = {חצות־הלילה של מר דוולנד},
titolo-eo     = {La noktomezo de S-ro Dowland},
titolo-xx     = {Mr Dowland’s Midnight},
ikono         = {🕛},
komponisto-he = {ג׳ון דוולנד},
komponisto-eo = {John Dowland},
komponisto-xx = {},
noto-he       = {ג׳ון דוולנד (1563–1626) היה מלחין ונגן לאוטה מתקופת הרנסנס. את מרבית המוסיקה שכתב הוא כתב ללאוטה, סולו או בליווי קולי או כלי; ככלל גם המוסיקה וגם המילים מלנכוליות (אחת היצירות המפורסמות שלו נקראת \Lat{Semper Dowland, semper dolens} „תמיד דוולנד, תמיד דואב”…). כאן עיבדתי את הקול הראשון של „חצות־הלילה של מר דוולנד”, מתוך הטבלטורה שבספר הלאוטה של מרגרט בורד.\pnurletikedo{gerbode.net/sources/spencer_private_library/margaret_board_lute_book/pdf/086_midnight_dowland.pdf}{boardmidnight}.},
noto-eo       = {John Dowland estis compnisto kaj ludisto de liuto de la Renesanco. La plejparto de gia verkaro estas skribita por liuto, kaj sole kaj kun voĉa aŭ instrumenta akompano; ĝenerale, kaj la muziko kaj la kantoteksto estas melanĥoliaj (unu le gia plej famaj muzikaĵoj estas nomata «Semper Dowland, semper dolens» (Ĉiam Dowland, ĉiam malĝoja)…). Tieĉi mi aranĝis la unua voĉo de «La noktomezo de S-ro Dowland» de la tabulaturo el la Libro de Liuto de Margaret Board.\pnurlref{gorkogo}},
]{midnight}

\begin{tabulaturo}
	\takto{}{\d\z\z\e\f\z\d\z}{}\par
	\takto{}{\f\z\g\z\e\z\cis\z}{\x}\par
	\takto{}{\d\z\z\e\f\z\e\z}{}\par
	\takto{}{\f\z\g\z\a\z\z\z}{\x}\par
	\takto{}{\d\z\d\e\f\z\d\z}{}\par
	\takto{}{\f\z\g\z\e\z\cis\z}{\x}\par
	\takto{}{\d\z\d\e\f\z\e\z}{}\par
	\takto{}{\d\e\f\g\a\z\z\z}{\r{0}{1}}\newpage
	\takto{}{\e\z\C\z\g\z\a\z}{}\par
	\takto{}{\bes\z\a\g\f\z\g\z}{\x}\par
	\takto{}{\a\z\g\a\g\z\f\z}{}\par
	\takto{}{\e\z\e\z\d\z\a\bes}{\x}\par
	\takto{}{\C\z\bes\a\g\f\g\a}{}\par
	\takto{}{\bes\z\a\g\f\e\f\g}{\x}\par
	\takto{}{\a\z\g\a\bes\a\g\f}{}\par
	\takto{}{\e\d\e\z\d\z\z\z}{\R{1}{0}}\par
\end{tabulaturo}
