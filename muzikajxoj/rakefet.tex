\muzikajxo[
titolo-he     = {הרקפת},
titolo-eo     = {La ciklameno},
titolo-xx     = {},
ikono         = {⚘},
komponisto-he = {עממי},
komponisto-eo = {\emph{popola}},
komponisto-xx = {},
noto-he       = {שיר ילדים שכתב המשורר והסופר לוין קיפניס למנגינה, עממית, של שיר שחיבר זלמן שניאור ביידיש ובעברית, „מאַרגאַריטקעלעך”~/ „מרגניות”. הרקע לכתיבת השיר הוא ביקורו של קיפניס במטולה בחורף 1921. בערב אחד ראה חבורת נערים ונערות שרים יחד, וכאשר ביקש לבקר במפל התנור פנתה אליו בת־שבע, המוזכרת בשיר, והציעה ללוות אותו. קיפניס התרגש מהמקום, מהאווירה ומהרקפות והחליט לכתוב את השיר.\pnurl{zemer.co.il/song.asp?id=2751}\,\pnurl{he.wikipedia.org/wiki/רקפת_(שיר)}\,\pnurl{wildflowers.co.il/hebrew/plant.asp?ID=61}},
noto-eo       = {Infana kanto, kiun verkis la poeto kaj verkisto Levin Kipnis laŭ la popola melodio de la kanto «Margaritkeleĥ» (Anagaloj), kiun Zalman Ŝneur verkis en la jida kaj la hebrea lingvoj. La fono de verki de la kanto estas vizito de Kipnis en Metula en la vintro de 1921. Unu vespere, gi vidis areton de junuloj kantantaj kune, kaj kiam gi demandis pri alveturi al la akvofalo Tanur, alparolis al gin Bat-Ŝeva, kiu estas menciita en la kantoteksto, kaj proponis kunveturi kun gi. Per gia impreso de la loko, la medio kaj la ciklamenoj gi decidiĝis verki la kanton.},
]{rakefet}

\begin{tabulaturo}
	\takto{}{\w\w\e}{\x}
	\takto{}{\a\a\a}{\x}\par
	\takto{}{\C\C\C}{\x}
	\takto{}{\b\b\b}{\x}\par
	\takto{}{\a\e\e}{\x}
	\takto{}{\g\g\g}{\x}\par
	\takto{}{\g\a\f}{\x}
	\takto{}{\e\z\z}{\x}\par
	\takto{}{\w\w\e}{\x}
	\takto{}{\C\a\g}{\x}\par
	\takto{}{\f\e\d}{\x}
	\takto{}{\d\f\a}{\x}\par
	\takto{}{\a\g\f}{\x}
	\takto{}{\e\dis\e}{\x}\par
	\takto{}{\C\C\b}{\x}
	\takto{}{\a\z\z}{\X}
\end{tabulaturo}
