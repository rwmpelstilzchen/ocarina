%⚙ https://en.wikipedia.org/wiki/Tourdion

\muzikajxo[
titolo-he     = {טורדיון},
titolo-eo     = {Tordiono},
titolo-xx     = {Tordion},
ikono         = {🔀},
komponisto-he = {אלמוני/ת},
komponisto-eo = {\emph{sennoma}},
komponisto-xx = {},
noto-he       = {טורדיון (מ־\Lat{tordre} בצרפתית, „לפתל”) הוא ריקוד, במקצב משולש, שהיה נפוץ בחלקים של אירופה בתקופת הרנסנס; דומה לריקוד גליארד אבל מהיר וחלק ממנו. הטורדיון המפורסם ביותר כיום הוא זה שעובד כאן; המלחין לא ידוע, והתווים הודפסו בשנת 1530 על־ידי המדפיס פייר אטניין\pnurl{he.wikipedia.org/wiki/פייר_אטניין} כחלק מ„לה מגדלנה”.},
noto-eo       = {Tordiono (el «tordre» (tordi) en la franca lingvo) estas tritakta danco, kiu estis disvastiĝinta en partoj de Eŭropo en la Renesanco; ĝi estas simila al la galjardo, sed estas pli rapida kaj pli glata. Nuntempe, la plej fama tordiono estas tiu, kiu estas aranĝita tie ĉi; la komponisto estas nekonata, kaj la muzikaĵo estis presita en 1530 per la presisto Pierre Attaignant\pnurl{eo.wikipedia.org/wiki/Pierre_Attaignant}, kiel parto de «La Magdalena».},
]{tordion}

\begin{tabulaturo}
	\takto{}{\d\e\f\g\f\e}{\x}\par
	\takto{}{\d\z\z\e\f\g}{\x}\par
	\takto{}{\a\g\f\f\g\e}{\x}\par
	\takto{}{\f\z\e\d\c\z}{\x}\par
	\takto{}{\d\e\f\g\f\e}{\x}\par
	\takto{}{\d\z\f\z\e\z}{\x}\par
	\takto{}{\d\z\z\z\c\z}{\x}\par
	\takto{}{\d\z\z\z\z\z}{\r{1}{1}}\newpage
	\takto{}{\a\z\z\g\a\bes}{\x}\par
	\takto{}{\a\z\z\z\a\z}{\x}\par
	\takto{}{\C\bes\a\g\f\e}{\x}\par
	\takto{}{\f\z\z\e\d\z}{\x}\par
	\takto{}{\a\z\z\g\a\bes}{\x}\par
	\takto{}{\a\z\g\f\e\z}{\x}\par
	\takto{}{\d\z\z\z\c\z}{\x}\par
	\takto{}{\d\z\z\z\z\z}{\R{1}{0}}
\end{tabulaturo}
