\muzikajxo[
titolo-he     = {לשדה יצוא יצאתי},
titolo-eo     = {Barieroj},
titolo-xx     = {{\hebrewfont באַריקאַדן}},
ikono         = {🐎},
komponisto-he = {עממי},
komponisto-eo = {\emph{popola}},
komponisto-xx = {},
noto-he       = {מקור הלחן לא ידוע, כמו גם מקור המילים העבריות. לאותו הלחן קיים גם שיר יידי, באַריקאַדן „בריקדות”\pnurletikedo{yidlid.org/chansons/barikadn/}{barikadn}, פרי עטו של שמערקע קאַטשערגינסקי\pnurl{he.wikipedia.org/wiki/שמריהו_קצ'רגינסקי} (1908–1954), המתאר את התקוממות פועלים בלודז׳; ביצוע של השיר מ־1948 בידי קאַטשערגינסקי ורות רובין\pnurletikedo{ololo.fm/search/Shmerke+Kacherginsky+\%28And+Ruth+Rubin\%29/Barikadn+1948}{barikadn1948}, וביצוע מודרני בידי הלהקה \Lat{The Klezmatics}\pnurletikedo{youtu.be/h7FZ-hvPRx8}{barikadnklezmatics}.},
noto-eo       = {La origino de la melodio aŭ la hebrea kantoteksto estas nekonata. La sama melodio havas ankaŭ, jida, kantoteksto, \emph{Barikadn} «Barieroj», far Ŝmerke Kaĉerginski (1908–1954), kiu priskribas la laborista popolleviĝo en Lodzo; kantita per Kaĉerginski kaj Rut Rubin\pnurlref{barikadn1948}, kaj per la muzikgrupo \emph{The Klezmatics}\pnurlref{barikadnklezmatics}.}
]{sade}

\begin{tabulaturo}
	\takto{}{\d\a\a\a\a\a\a\a}{\x}\par
	\takto{}{\a\g\g\f\a\z\a\w}{\x}\par
	\takto{}{\d\a\a\a\a\a\a\a}{\x}\par
	\takto{}{\a\g\g\f\f\z\f\w}{\x}\par
	\takto{}{\d\g\d\g\f\f\f\w}{\x}\par
	\takto{}{\f\e\e\d\f\z\f\w}{\x}\par
	\takto{}{\d\g\d\g\f\f\f\w}{\x}\par
	\takto{}{\f\e\e\d\d\z\d\z}{\X}
\end{tabulaturo}
