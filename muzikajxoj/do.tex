\muzikajxo[
titolo-he     = {דו־רה־מי},
titolo-eo     = {Do-Re-Mi},
titolo-xx     = {},
ikono         = {🐻},
komponisto-he = {ריצ׳רד רוג׳רז},
komponisto-eo = {Richard Rodgers},
komponisto-xx = {},
noto-he       = {שיר מתוך המחזמר והסרט „צלילי המוסיקה” שיצא בשנת 1959. במסגרת הסיפור, האומנת מריה משתמשת בשיר כדי ללמד את הילדים התווים. השיר כתוב בסולם דו מזו׳ר, והוא מתאר סולפז׳ בסיסי בעל-פה.\pnurl{he.wikipedia.org/wiki/דו-רה-מי_(שיר)}},
noto-eo       = {Kanto el la muzikteatraĵo kaj filmo «La Sono de Muziko», kiu aperis en 1959. En la rakonto, la vartisto Maria uzas la kanton por lernigi la muziknotojn al la infanoj. La kanto estas komponita en la gamo de C-maĵoro, kaj ĝi priskribas solfeĝon voĉe.},
]{do}

\begin{tabulaturo}
	\takto{}{\c\z\z\d\e\z\z\c}{\x}\par
	\takto{}{\e\z\c\z\e\z\z\z}{\x}\par
	\takto{}{\d\z\z\e\f\f\e\d}{\x}\par
	\takto{}{\f\z\z\z\z\z\z\z}{\x}\par
	\takto{}{\e\z\z\f\g\z\z\e}{\x}\par
	\takto{}{\g\z\e\z\g\z\z\z}{\x}\par
	\takto{}{\f\z\z\g\a\a\g\f}{\x}\par
	\takto{}{\a\z\z\z\z\z\z\z}{\x}\par
	\takto{}{\g\z\z\c\d\e\f\g}{\x}\newpage
	\takto{}{\a\z\z\z\z\z\z\z}{\x}\par
	\takto{}{\a\z\z\d\e\fis\g\a}{\x}\par
	\takto{}{\b\z\z\z\z\z\z\z}{\x}\par
	\takto{}{\b\z\z\e\fis\gis\a\b}{\x}\par
	\takto{}{\C\z\z\z\z\z\C\b}{\x}\par
	\takto{}{\a\z\f\z\b\z\g\z}{\x}\par
	\takto{}{\C\z\g\z\e\z\d\z}{\r{1}{0}}\par
	\takto{}{\c\z\z\z\z\z\z\z}{\X}
\end{tabulaturo}
