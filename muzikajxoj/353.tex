\muzikajxo[
titolo-he     = {קנטיגה~353},
titolo-eo     = {Kantigo~353},
titolo-xx     = {Quen a omagen da Virgen},
ikono         = {♍},
komponisto-he = {עממי~/ אלפונזו העשירי},
komponisto-eo = {\emph{popola} / Alfonso la 10-a},
komponisto-xx = {},
noto-he       = {\Lat{Cantigas de Santa Maria} „הקנטיגות של מרים הקדושה” הוא אוסף של 420 שירים בקול יחיד בגליסיאנית ימיביניימית: תווים ומילים. זהו האוסף הגדול ביותר מסוגו. כל השירים מאזכרים את מרים, וכל שיר עשירי הוא מסוג „המנון”.},
noto-eo       = {\emph{Cantigas de Santa Maria} «La Kantigoj de Sankta Maria» estas aro de 420 monofonaj kantoj en mezepoka galegalingvo: notoj kaj vortoj. Tio estas la plej granda tia aro. Ĉiuj kantoj mencias Maria-on, kaj ĉiu deka kanto estas \emph{himno}.},
]{353}

\begin{tabulaturo}
	\takto{}{\d\z\a\z\a\z\z\g}{\x}\par
	\takto{}{\f\z\f\g\a\z\z\d}{\x}\par
	\takto{}{\d\z\a\z\a\z\z\g}{\x}\par
	\takto{}{\f\e\d\z\f\z\z\z}{\x}\par
	\takto{}{\d\z\a\z\a\z\z\g}{\x}\par
	\takto{}{\f\z\f\g\a\z\z\d}{\x}\par
	\takto{}{\d\z\a\z\a\z\z\g}{\x}\par
	\takto{}{\f\e\d\c\d\z\z\z}{\x}\newpage
	\takto{}{\a\z\b\z\g\z\z\a}{\x}\par
	\takto{}{\b\z\C\b\a\z\z\g}{\x}\par
	\takto{}{\a\z\a\b\a\z\z\g}{\x}\par
	\takto{}{\f\z\g\z\f\z\z\z}{\x}\par
	\takto{}{\a\z\b\z\g\z\z\a}{\x}\par
	\takto{}{\b\z\C\b\a\z\z\g}{\x}\par
	\takto{}{\a\z\a\b\a\z\z\g}{\x}\par
	\takto{}{\f\e\d\c\d\z\z\z}{\X}\par
\end{tabulaturo}
