\hesection{הקדמה}

הרעיון לחוברת הזאת נולד אחרי שקניתי עם ראם אוֹקָרִינָה\footnote{כלי־נשיפה סגלגל עם פומית בולטת החוצה; בדרך־כלל גודלו קטן והוא עשוי מחרס. קיימת שונות רבה בין האוקרינות השונות: הן בצורה, הן בגודל (מזעירות ועד לענקיות), הן בחומר והן במספר החורים (בדרך־כלל מ־4 עד 12). מקור השם, \Lat{ocarina}, הוא מצורת ההקטנה של \Lat{oca} „אווז” בדיאלקט של בולוניה, שהיא עצמה מילה שמגיעה מצורת \Lat{avicula} „עוף קטן” של \Lat{avis} „עוף” בלטינית; בעברית זה יהיה משהו כמו „ציפורפורונת”…}: פרת משה רבינו חמודה כזאת עם ארבעה חורים, קצת כמו הלוגו של „פושישית”. האוקרינה הגיעה עם דף שבצדו האחד הסבר על האצבוע ובצדו השני שניים־שלושה שירים בטבלטורה דומה לזו שבחוברת הזאת. הדף אבד תוך דקה (ונמצא אחרי כמה חודשים…), כך שהיה צורך למצוא תחליף, ומכאן המרחק ליצירה של מקבילה אוקרינאית לְאוּקַלֵּיקַלּוּת היה קצר. מה זה „אוקליקלות”, אתם שואלים? חוברת עם מנגינות קלות לנגינה בְּאוּקוּלֶלֶה\footnote{כלי־פריטה קטנטן בעל ארבעה מיתרים ומנעד של שתי אוקטבות. גם כאן מקור השם בקטנטנוּת: \Lat{{\fontspec{Gentium}ʻ}ukulele} משמעו „פרעוש קופץ” בשפה ההוואיאית.} שהכנתי, במקור לאילאיל. כמו החוברת הזאת, גם בה המנגינות כתובות בטבלטורה (אם כי באופן טבעי כזו שהיא שונה באופיה, בגלל ההבדלים בין הכלים), גם היא זמינה להורדה באופן חופשי (ר׳ קישור בסוף ההקדמה) וקיימת חפיפה חלקית בין המנגינות כאן ושם.



\hesection{מה זה טבלטורה ואיך קוראים אותה?}

אנחנו רגילים לכתיבה של מוזיקה על־גבי חַמְשָׁה, כאשר גובה הצליל מסומן על־ידי גובה התו על־גבי החמשה ואורכו מסומן על־ידי צורת התו. הצורת הכתיבה הזאת מתאימה לכלים רבים (כלומר, היא לא ספציפית לכלי מסויים) ויש לה יתרונות תיאורטיים ופרקטיים. בטבלטורה, לעומת זאת, מה שמסומן הוא לא הגובה של הצליל, אלא אופן ההפקה שלו על כלי מסויים, מה שהופך אותה לנגישה יותר ללומדים צעירים, בגלל שנחסך מהם הצורך לתווך בין סימון מופשט לביצוע קונקרטי בכלי שבו הם מנגנים.

המנגינות שבחוברת הזאת כתובות באופן שמתאים לאוקרינה בת ארבעה חורים (או לאוקרינה בת שישה חורים אם סותמים את שני החורים שבחלק התחתון). מחזיקים אותה בשתי ידיים, כשבכל יד האצבע והאמה יכולות לסתום את שני החורים שבכל צד. כל צליל מסומן בטבלטורה על־ידי ריבוע שבו ארבעה חורים: הריבוע מייצג באופן סכימטי\footnote{במתכוון בחרתי בייצוג סכימטי ומינימליסטי. אם תחפשו \Lat{\texttt{ocarina tabs}} או \Lat{\texttt{ocarina tablature}} ת מצאו מגוון ייצוגים שהם אולי יותר ריאליסטיים, אבל הרבה פחות קריאים ונוחים.} את האוקרינה, כשהפיה נמצאת בחלק התחתון. חור לבן הוא פתוח, חור כהה סגור וחור מקווקו חצי־סגור. לדוגמה:
\begin{compactitem}
	\item אם כתוב \enliniatabulaturo{\c}, סותמים את כל ארבעת החורים ונושפים;
	\item אם כתוב \enliniatabulaturo{\gis}, סותמים באמה שביד ימין את החור הרחוק מהפה שבצד ימין ואצבע שביד שמאל את החור הקרוב אל הפה שביד שמאל ונושפים.
	\item אם כתוב \enliniatabulaturo{\C}, נושפים כשכל החורים פתוחים.
\end{compactitem}

אוקרינות שונות יכולות להיות מכוונים לגבהי־צליל שונים באותו האצבוע. אם האוקרינה מכוונת לדו, הנה סולם כרומטי על דו (מודגשים התווים בסולם דו מז׳ורי):

\begin{tabular}{cccccc}
	\c & \cis & \d & \dis & \e & \f\\
	\textbf{דו} &
	דו\symbolglyph{♯} / רה\symbolglyph{♭} &
	\textbf{רה} &
	רה\symbolglyph{♯} / מי\symbolglyph{♭} &
	\textbf{מי} &
	\textbf{פה}\\
	\fis & \g & \gis & \a & \ais & \b\\
	פה\symbolglyph{♯} / סול\symbolglyph{♭} &
	\textbf{סול} &
	סול\symbolglyph{♯} / לה\symbolglyph{♭} &
	\textbf{לה} &
	לה\symbolglyph{♯} / סי\symbolglyph{♭} &
	\textbf{סי}
\end{tabular}

דו באוקטבה שניה הוא \enliniatabulaturo{{\C}}. אם נושפים חלש כשכל החורים סתומים מופק צליל קצת נמוך יותר (\enliniatabulaturo{\B}), כך שעקרונית ניתן לנגן גם סי באוקטבה נמוכה. זהו, 13 וחצי צלילים שונים: די מדהים איזה מגוון של מנגינות אפשר לנגן בעזרת הצלילים האלה.

לכל ריבוע משך קבוע. \enliniatabulaturo{\z} תופס יחידה קצבית אחת שממשיכה את הקודמת, כך של־\enliniatabulaturo{\mbox{\LR{\c\z}}} משך כפול מל־\enliniatabulaturo{\c}. היתרון שבשיטה הזאת הוא כפול: פישוט של הקריאה עבור נגנים צעירים או מי שאין להם נסיון בקריאת תווים (אין תיווך של סימנים מופשטים), והנגשה של המבנה הקצבי של המנגינה באופן חזותי (אפשר בקלות לראות את התבניות החוזרות).

חזרות מסומנות ב־\enliniatabulaturo{\r{0}{1}} ו־\enliniatabulaturo{\r{1}{0}} (מקבילות ל־\symbolglyph{𝄆} ו־\symbolglyph{𝄇} בתיווי המקובל). כשיש הבדל בין החזרה הראשונה והשניה משמש סימון \symbolglyph{⌜} ממוספר מעל התיבות השונות.

רוצים לוודא שהבנתם? הנה ההתחלה של „יונתן הקטן”. תנסו לראות אם זה אכן מה שיוצא לכם כשאתם מנגנים:

\begin{samepage}
	\begin{LTR}
		\takto{}{\g\e\e\z}{\x}
		\takto{}{\f\d\d\z}{\x}\par
		\takto{}{\c\d\e\f}{\x}
		\takto{}{\g\g\g\z}{\x}
	\end{LTR}
\end{samepage}



\hesection{מידע נוסף ויצירת קשר}

%\begin{wrapfigure}[4]{l}{1.75cm}\vspace{-\baselineskip}\includegraphics[width=1.5cm]{../retejo.png}\end{wrapfigure}
האתר של החוברת הוא \url{http://xpr.digitalwords.net/ocarina}

(קוד \Lat{QR} בצד שמאל). באתר מידע נוסף וקישורים, כמו גם כל קבצי המקור של החוברת וקובץ מוכן להדפסה (אם תרצו ליצור עותק נוסף או לקרוא במחשב).

כל מה שעשיתי אני בחוברת הזאת משוחרר לחופשי (נחלת הכלל, \Lat{public domain}): אתם יכולים להעתיק, לשנות, להפיץ ולהשתמש בה כרצונכם.

בכל עניין, אל תהססו לפנות אלי בדואל (\url{ocarina@digitalwords.net}) או בטלפון (02-6419913). בפרט, אשמח לשמוע כל מה יש לכם להגיד על החוברת, ולקבל הצעות, שיפורים ותיקונים.

\vspace{\baselineskip}
~\hfill
\begin{minipage}{3cm}
	יודה רונן\\
	ירושלים 2015\\
\end{minipage}
