\subsection*{Antaŭparolo}

La ideo por tiu ĉi libreto naskiĝis post mi aĉetis okarinon\footnote{Ovala blovinstrumento kun elstaranta buŝaĵo; tipe estas malgranda kaj farita el ceramiko. Estas ega vario inter aliaj okarinoj: je grando (el etaj al grandegaj), je materio kaj je la nombro de truoj (tipe el 4 al 12). La etimologio de la nomo \emph{ocarina} «ansereto» estas el la diminutivo de \emph{oca} «ansero» en la dialekto de Bolonjo, kiu venas el diminutivo mem (\emph{avicula} «birdeto», el \emph{avis} «birdo» en la latina lingvo); sekve, ni povas esperantigi tion kiel «birdeteto»…} kun Re’em: beleta kokcinelo kun kvar truoj, simila al la markemblemo de Puŝiŝit. La okarino venis kun paĝo, ke sur unu de ĝia flankoj estis priskribo de la fingrado kaj sur la alia estis kelkaj kantoj en tabulaturo, kiu similas al tio, kiu estas en la ĉi tiu libreto. La paĝo perdiĝis post unu minuto (kaj estis retrovita post kelkaj monatoj…), nu ni devis havi alternativon: Okalina.


\subsection*{Kio estas tabulaturo, kaj kiel oni legas ĝin?}

Ni konas skribado de muziko per la kutima moderna okcidentuma notskribo, signifante la tonalto per grafika alto kaj la tondaŭro per la formo de notoj. Tiu muzika notacio konvenas multajn instrumentojn (tiu estas malspecifa), kaj havas multajn teoriajn kaj realajn utilojn. En tabulaturo, tamen, tio, kiu estast signifita, ne estas la tonalto, sed la maniero de sia farado per specifa instrumento. Tio faras la tabulaturon pli alirebla por junaj lernantoj.

La muzikaĵoj en tiu ĉi libreto estas skribitaj je meniero, kiu konvenas kvar-truan okarinan (aŭ ses-truan okarinan, se oni fermas la du malsuprajn truojn). Oni tenas ĝin kun du manoj, kaj ludas kun la montrofingroj kaj la mezfingroj. Ĉiu tono estas signifita en la tabulaturo per kvadrato (la okarino) kun kvar kvadratetoj (la truoj); la buŝaĵo estast malsupre. Blanka truo estas malfermita, nigra truo estas fermita, kaj truo, kiu pleniĝas per diagonalaj linioj, estas mez-fermita. Ekzemple,
\begin{compactitem}
	\item Legante \enliniatabulaturo{\c}, oni devas blovi post fermi ĉiujn kvar truojn.
	\item Legante \enliniatabulaturo{\gis}, oni devas blovi post fermi la truon per la dekstra mezfingro kaj la truon per la maldekstra montrofingro.
	\item Legante \enliniatabulaturo{\C}, oni devas blovi post malfermi la ĉiujn kvar truojn.
\end{compactitem}
