\enhavo
\hesection{תוכן עניניים}

\parbox[t]{0.46\textwidth}{\noindent\setRL%
	עבור כל מנגינה כתבתי פסקה קצרה של רקע. מקורות המידע העיקריים שהשתמשתי בהם היו ויקיפדיה ואתר זמרשת\pnurletikedo{zemer.co.il}{zemereshet} (לשירים העבריים). תודה לעורכים בשני האתרים האלה. \symbolglyph{☺}
}\hfill
\parbox[t]{0.50\textwidth}{\noindent\setLR%
	\EO{%
		Pri ĉiu melodio mi skribis mallongan alineon. Mia primaj informfontoj estis Vikipedio kaj la retejo «Zemereŝet\pnurlref{zemereshet}» (por la hebreaj kantoj). Dankon por la redaktoroj de tiu du retejoj. \symbolglyph{☺}
	}
}

%%\muzikajxo[
titolo-he     = {אוֹדָה לשמחה},
titolo-eo     = {Odo al ĝojo},
titolo-xx     = {An die Freude},
ikono         = {☺},
komponisto-he = {לודוויג ון בטהובן},
komponisto-eo = {Ludwig van Beethoven},
komponisto-xx = {},
noto-he       = {אחד משיריו המפורסמים ביותר של המשורר פרידריך שילר, וכן שמו של העיבוד לאודה המופיע בפרק הרביעי והאחרון של הסימפוניה התשיעית מאת לודוויג ואן בטהובן. האודה מדגישה, בפאתוס רב, את האידאל של חברת בני־אדם שוויונית שבה שוררות שמחה ואחווה.\pnurl{he.wikipedia.org/wiki/האודה_לשמחה}},
noto-eo       = {Unu el la plej famaj poemoj de la poeto Friedrich Schiller. Oni konas ĝin plejparte ĉar ĝi estas la teksto de la ĥoroj en la kvara movimento el la Betovena Naŭa Simfonio. Tiu odo honoras la homan fratecon, ĝoje repacigita kun la Dio kreinto.\pnurl{eo.wikipedia.org/wiki/Odo_al_\%C4\%9Cojo}},
]
{freude}

\begin{tabulaturo}
	\takto{}{\e\z\e\z\f\z\g\z}{\x}\par
	\takto{}{\g\z\f\z\e\z\d\z}{\x}\par
	\takto{}{\c\z\c\z\d\z\e\z}{\x}\par
	\takto{}{\e\z\z\d\d\z\z\z}{\x}\par
	\takto{}{\e\z\e\z\f\z\g\z}{\x}\par
	\takto{}{\g\z\f\z\e\z\d\z}{\x}\par
	\takto{}{\c\z\c\z\d\z\e\z}{\x}\par
	\takto{}{\d\z\z\c\c\z\z\z}{\x}\newpage
	\takto{}{\d\z\d\z\e\z\c\z}{\x}\par
	\takto{}{\d\z\e\f\e\z\c\z}{\x}\par
	\takto{}{\d\z\e\f\e\z\d\z}{\x}\par
	\takto{}{\c\z\d\z\g\z\z\z}{\x}\par
	\takto{}{\e\z\e\z\f\z\g\z}{\x}\par
	\takto{}{\g\z\f\z\e\z\d\z}{\x}\par
	\takto{}{\c\z\c\z\d\z\e\z}{\x}\par
	\takto{}{\d\z\z\c\c\z\z\z}{\X}\par
\end{tabulaturo}

%%\muzikajxo[
titolo-he     = {אי־שם מעבר לקשת},
titolo-eo     = {Trans ĉielarko},
titolo-xx     = {Over the Rainbow},
ikono         = {🌈},
komponisto-he = {הרולד ארלן},
komponisto-eo = {Harold Arlen},
komponisto-xx = {},
]{rainbow}

\begin{tabulaturo}
	\takto{}{\c\z\z\z\C\z\z\z}{\x}\par
	\takto{}{\b\z\g\a\b\z\C\z}{\x}\par
	\takto{}{\c\z\z\z\a\z\z\z}{\x}\par
	\takto{}{\g\z\z\z\z\z\z\z}{\x}\par
	\takto{}{\c\z\z\z\f\z\z\z}{\x}\par
	\takto{}{\e\z\c\d\e\z\f\z}{\x}\par
	\takto{}{\d\z\B\c\d\z\e\z}{\x}\par
	\takto{1}{\c\z\z\z\z\z\z\z}{\R{1}{0}}\par
	\takto{2}{\c\z\z\z\z\z\z\g}{\x}\newpage
	\takto{}{\e\g\e\g\e\g\e\g}{\x}\par
	\takto{}{\f\g\f\g\f\g\f\g}{\x}\par
	\takto{}{\a\z\z\z\a\z\z\z}{\x}\par
	\takto{}{\z\z\z\z\z\z\z\g}{\x}\par
	\takto{}{\e\g\e\g\e\g\e\g}{\x}\par
	\takto{}{\fis\a\fis\a\fis\a\fis\a}{\x}\par
	\takto{}{\b\z\z\z\b\z\z\z}{\x}\par
	\takto{}{\z\z\z\z\z\z\z\z}{\x\reludu}\par
\end{tabulaturo}

%%\muzikajxo[
titolo-he     = {אמא יקרה לי},
titolo-eo     = {Mia kara patrino},
titolo-xx     = {},
ikono         = {♥},
komponisto-he = {נחום נרדי},
komponisto-eo = {Naĥum Nardi},
komponisto-xx = {},
noto-he       = {מילים: משה דפנא.\pnurl{zemer.co.il/song.asp?id=920}},
noto-eo       = {Kantoteksto: Moŝe Dafna.},
]{ima}

\begin{tabulaturo}
\takto{}{\d\f\e\f}{\x}
\takto{}{\e\z\d\z}{\x}\par
\takto{}{\f\z\g\z}{\x}
\takto{}{\a\z\z\z}{\r{1}{1}}\par
\takto{}{\a\a\a\f}{\x}
\takto{}{\g\g\g\z}{\x}\par
\takto{}{\f\f\f\d}{\x}
\takto{}{\e\e\e\a}{\x}\par
\takto{}{\f\z\e\z}{\x}
\takto{}{\d\z\z\a}{\x}\par
\takto{}{\f\z\e\z}{\x}
\takto{}{\d\z\z\z}{\R{1}{0}}
\end{tabulaturo}

%%\muzikajxo[
titolo-he     = {אני עומדת במעגל},
titolo-eo     = {Mi staras en la cirklo},
titolo-xx     = {},
ikono         = {○},
komponisto-he = {עממי},
komponisto-eo = {\emph{popola}},
komponisto-xx = {},
noto-he       = {מקורותיהם הן של המילים והן של הלחן לא ברורים, וישנן כמה גרסאות להתגלגלותם.\pnurl{zemer.co.il/song.asp?id=976}},
noto-eo       = {La originoj kaj de la kantoteksto kaj de la melodio estas neklaraj.},
]{bamaagal}

\begin{tabulaturo}
	\takto{}{\c}{\x}\par
	\takto{}{\f\f\f\f}{\x}
	\takto{}{\e\f\g\f}{\x}\par
	\takto{}{\e\e\e\d}{\x}
	\takto{}{\c\z\z\c}{\x}\par
	\takto{}{\g\g\g\g}{\x}
	\takto{}{\g\a\bes\g}{\x}\par
	\takto{}{\C\C\C\a}{\x}
	\takto{}{\f\z\z\c}{\r{0}{1}}\par
	\takto{}{\f\z\a\f}{\x}
	\takto{}{\d\z\c\c}{\x}\par
	\takto{}{\g\z\a\g}{\x}
	\takto{\alt{1}}{\f\a\C\c}{\r{1}{0}}\par
	\takto{\alt{2}}{\f\z\z\z}{\X}
\end{tabulaturo}

%%\muzikajxo[
titolo-he     = {דו־רה־מי},
titolo-eo     = {Do-Re-Mi},
titolo-xx     = {},
ikono         = {🐻},
komponisto-he = {ריצ׳רד רוג׳רז},
komponisto-eo = {Richard Rodgers},
komponisto-xx = {},
]{do}

\begin{tabulaturo}
	\takto{}{\c\z\z\d\e\z\z\c}{\x}\par
	\takto{}{\e\z\c\z\e\z\z\z}{\x}\par
	\takto{}{\d\z\z\e\f\f\e\d}{\x}\par
	\takto{}{\f\z\z\z\z\z\z\z}{\x}\par
	\takto{}{\e\z\z\f\g\z\z\e}{\x}\par
	\takto{}{\g\z\e\z\g\z\z\z}{\x}\par
	\takto{}{\f\z\z\g\a\a\g\f}{\x}\par
	\takto{}{\a\z\z\z\z\z\z\z}{\x}\par
	\takto{}{\g\z\z\c\d\e\f\g}{\x}\newpage
	\takto{}{\a\z\z\z\z\z\z\z}{\x}\par
	\takto{}{\a\z\z\d\e\fis\g\a}{\x}\par
	\takto{}{\b\z\z\z\z\z\z\z}{\x}\par
	\takto{}{\b\z\z\e\fis\gis\a\b}{\x}\par
	\takto{}{\C\z\z\z\z\z\C\b}{\x}\par
	\takto{}{\a\z\f\z\b\z\g\z}{\x}\par
	\takto{}{\C\z\g\z\e\z\d\z}{\r{1}{0}}\par
	\takto{}{\c\z\z\z\z\z\z\z}{\X}
\end{tabulaturo}

%%\muzikajxo[
titolo-he     = {דוגית נוסעת},
titolo-eo     = {Boato moviĝas},
titolo-xx     = {},
ikono         = {⛵},
komponisto-he = {לב שוורץ},
komponisto-eo = {Lev Ŝvarc},
komponisto-xx = {},
noto-he       = {את המילים בנוסח העברי כתב נתן יונתן, והקדיש אותן בשנת 1944 (ככל הנראה זמן לא רב לאחר חיבור השיר) לילדי קיבוץ שפיים (במילותיו: „שיר קטן~— לכל הילדים החביבים”). את המנגינה כתב לב שוורץ עבור הסרט „ילדותו של גּוֹרְקִי”\pnurletikedo{youtu.be/o_741ZmF0c8}{gorkogo} (\Rus{Детство Горького}, 1938), אולי בהתבסס על ניגון יהודי.\pnurl{zemer.co.il/song.asp?id=1595}},
noto-eo       = {La hebrean kantotekston skribis Natan Jonatan, kaj ĝin dediĉis en 1944 (kredeble ne longe post la verkado) al la infanoj de kibuco Ŝfajim (en gia vortoj: «kanteto~— por ĉiuj la plaĉemaj infanoj»). La melodion komponis Lev Ŝvarc por la filmo «La Infanaĝo de Gorki»\pnurlref{gorkogo} (\Rus{Детство Горького}, 1938), eble depruntinte de juda melodio.},
]{dugit}

\begin{tabulaturo}
	\takto{}{\c\z\ees\c}{\x}
	\takto{}{\f\z\aes\z}{\x}\par
	\takto{}{\g\aes\g\f}{\x}
	\takto{}{\ees\z\c\z}{\x}\par
	\takto{}{\c\z\ees\c}{\x}
	\takto{}{\f\z\aes\z}{\x}\par
	\takto{}{\g\aes\g\f}{\x}
	\takto{}{\C\z\z\z}{\r{0}{1}}\par
	\takto{}{\C\z\bes\aes}{\x}
	\takto{}{\bes\z\ees\z}{\x}\par
	\takto{}{\aes\z\g\f}{\x}
	\takto{}{\ees\z\c\z}{\x}\par
	\takto{}{\c\z\ees\c}{\x}
	\takto{}{\f\z\aes\z}{\x}\par
	\takto{}{\g\aes\g\f}{\x}
	\takto{1}{\C\z\z\z}{\r{1}{0}}\par
	\takto{2}{\c\z\z\z}{\X}
\end{tabulaturo}

%%\muzikajxo[
titolo-he     = {היום יום־הולדת},
titolo-eo     = {Hodiaŭ estas naskiĝtago},
titolo-xx     = {},
ikono         = {🎈},
komponisto-he = {ורדה גלבוע},
komponisto-eo = {Varda Gilboa},
komponisto-xx = {},
]{jomuledet}

\begin{tabulaturo}
	\takto{}{\e\z\e\e\e\z\d\c}{\x}\par
	\takto{}{\f\z\f\f\f\z\e\d}{\x}\par
	\takto{}{\g\z\g\a\g\f\e\d}{\x}\par
	\takto{1}{\e\z\f\z\g\z\z\z}{\r{1}{0}}\par
	\takto{2}{\e\z\d\z\c\z\z\z}{\r{0}{1}}\par
	\takto{}{\a\z\a\a\a\z\g\f}{\x}\par
	\takto{}{\g\z\g\g\C\z\g\g}{\x}\par
	\takto{}{\f\z\f\f\g\f\e\d}{\x}\par
	\takto{1}{\e\z\f\z\g\z\z\z}{\r{1}{0}}\par
	\takto{2}{\e\z\d\z\c\z\z\z}{\X}
\end{tabulaturo}

%%\muzikajxo[
titolo-he     = {הנה באה הרכבת},
titolo-eo     = {Jen la traĵno venas},
titolo-xx     = {},
ikono         = {🚂},
komponisto-he = {עממי},
komponisto-eo = {Popola},
komponisto-xx = {},
]{rakevet}

\begin{tabulaturo}
	\takto{}{\c\c\c\c}{\x}
	\takto{}{\e\c\c\c}{\x}\par
	\takto{}{\d\d\g\g}{\x}
	\takto{}{\e\c\c\c}{\x}\par
	\takto{}{\a\f\a\f}{\x}
	\takto{}{\e\e\e\z}{\x}\par
	\takto{}{\d\f\e\d}{\x}
	\takto{}{\c\e\g\z}{\x}\par
	\takto{}{\a\f\a\f}{\x}
	\takto{}{\e\e\e\z}{\x}\par
	\takto{}{\d\f\e\d}{\x}
	\takto{}{\c\c\c\z}{\X}
\end{tabulaturo}

%%\muzikajxo[
titolo-he     = {הרקפת},
titolo-eo     = {La ciklameno},
titolo-xx     = {},
ikono         = {⚘},
komponisto-he = {עממי},
komponisto-eo = {\emph{popola}},
komponisto-xx = {},
noto-he       = {שיר ילדים שכתב המשורר והסופר לוין קיפניס למנגינה, עממית, של שיר שחיבר זלמן שניאור ביידיש ובעברית, „מאַרגאַריטקעלעך”~/ „מרגניות”. הרקע לכתיבת השיר הוא ביקורו של קיפניס במטולה בחורף 1921. בערב אחד ראה חבורת נערים ונערות שרים יחד, וכאשר ביקש לבקר במפל התנור פנתה אליו בת־שבע, המוזכרת בשיר, והציעה ללוות אותו. קיפניס התרגש מהמקום, מהאווירה ומהרקפות והחליט לכתוב את השיר.\pnurl{zemer.co.il/song.asp?id=2751}\,\pnurl{he.wikipedia.org/wiki/רקפת_(שיר)}\,\pnurl{wildflowers.co.il/hebrew/plant.asp?ID=61}},
noto-eo       = {Infana kanto, kiun verkis la poeto kaj verkisto Levin Kipnis laŭ la popola melodio de la kanto «Margaritkeleĥ» (Anagaloj), kiun Zalman Ŝneur verkis en la jida kaj la hebrea lingvoj. La fono de verki de la kanto estas vizito de Kipnis en Metula en la vintro de 1921. Unu vespere, gi vidis areton de junuloj kantantaj kune, kaj kiam gi demandis pri alveturi al la akvofalo Tanur, alparolis al gin Bat-Ŝeva, kiu estas menciita en la kantoteksto, kaj proponis kunveturi kun gi. Per gia impreso de la loko, la medio kaj la ciklamenoj gi decidiĝis verki la kanton.},
]{rakefet}

\begin{tabulaturo}
	\takto{}{\w\w\e}{\x}
	\takto{}{\a\a\a}{\x}\par
	\takto{}{\C\C\C}{\x}
	\takto{}{\b\b\b}{\x}\par
	\takto{}{\a\e\e}{\x}
	\takto{}{\g\g\g}{\x}\par
	\takto{}{\g\a\f}{\x}
	\takto{}{\e\z\z}{\x}\par
	\takto{}{\w\w\e}{\x}
	\takto{}{\C\a\g}{\x}\par
	\takto{}{\f\e\d}{\x}
	\takto{}{\d\f\a}{\x}\par
	\takto{}{\a\g\f}{\x}
	\takto{}{\e\dis\e}{\x}\par
	\takto{}{\C\C\b}{\x}
	\takto{}{\a\z\z}{\X}
\end{tabulaturo}

%%\muzikajxo[
titolo-he     = {השקדיה פורחת},
titolo-eo     = {La migdalarbo floras},
titolo-xx     = {},
ikono         = {🌸},
komponisto-he = {מנשה רבינא},
komponisto-eo = {Menaŝe Ravina},
komponisto-xx = {},
]{ŝkedija}

\begin{tabulaturo}
	\takto{}{\c}{\x}\par
	\takto{}{\c\c\c\e\g\z\e\e}{\x}\par
	\takto{}{\g\g\a\a\g\z\e\z}{\x}\par
	\takto{}{\c\d\e\c\f\a\g\z}{\x}\par
	\takto{}{\g\g\a\g\e\g\c\z}{\x}\par
	\takto{}{\C\C\a\a\g\z\e\z}{\x}\par
	\takto{}{\f\f\e\d\g\z\z\z}{\x}\par
	\takto{}{\C\C\a\a\g\z\e\z}{\x}\par
	\takto{}{\f\f\e\d\c\z\z\z}{\X}
\end{tabulaturo}

%%\muzikajxo[
titolo-he     = {וילסון׳ז ויילד},
titolo-eo     = {Wilson’s Wilde},
titolo-xx     = {},
ikono         = {❦},
komponisto-he = {אלמוני/ת},
komponisto-eo = {\emph{sennoma}},
komponisto-xx = {},
noto-he       = {הקול הראשון של הנעימה, מתקופת הרנסנס, מתוך הטבלטורה שבספר הלאוטה של סמפסון.\pnurletikedo{cs.dartmouth.edu/~wbc/tab-serv/tablature.cgi?Misc_English/WilsonsWilde.pdf}{wilsonsliuto}},
noto-eo       = {La unua voĉo de la renesanca muzikaĵo de la tabulaturo el la Libro de la Liuto de Sampsono.\pnurlref{wilsonsliuto}},
]
{wilsons}

\begin{tabulaturo}
	\takto{}{\g\z}{\x}\par
	\takto{}{\C\z\z\z\g\z}{\x}\par
	\takto{}{\e\z\c\z\f\z}{\x}\par
	\takto{}{\e\z\z\d\c\z}{\x}\par
	\takto{}{\d\z\z\z\g\z}{\x}\par
	\takto{}{\C\z\z\z\g\z}{\x}\par
	\takto{}{\e\z\c\z\f\z}{\x}\par
	\takto{}{\d\z\z\c\d\z}{\x}\par
	\takto{}{\c\z\z\z\g\z}{\x}\newpage
	\takto{}{\C\z\z\z\g\z}{\x}\par
	\takto{}{\e\z\c\z\f\z}{\x}\par
	\takto{}{\e\z\f\e\d\c}{\x}\par
	\takto{}{\d\z\z\z\g\z}{\x}\par
	\takto{}{\C\z\z\z\g\z}{\x}\par
	\takto{}{\e\z\c\d\e\f}{\x}\par
	\takto{}{\d\c\B\c\d\B}{\x}\par
	\takto{}{\c\z\z\z\z\z}{\x\x}\newpage
	\takto{}{\e\z\z\f\g\z}{\x}\par
	\takto{}{\g\z\z\a\g\z}{\x}\par
	\takto{}{\a\z\z\b\C\z}{\x}\par
	\takto{}{\b\z\z\z\g\z}{\x}\par
	\takto{}{\e\d\e\f\g\z}{\x}\par
	\takto{}{\g\f\g\a\g\z}{\x}\par
	\takto{}{\a\g\a\b\C\z}{\x}\par
	\takto{}{\b\z\z\z\g\z}{\x\x}\newpage
	\takto{}{\e\z\g\z\z\z}{\x}\par
	\takto{}{\f\z\a\z\z\z}{\x}\par
	\takto{}{\e\z\g\z\z\z}{\x}\par
	\takto{}{\d\z\f\z\z\z}{\x}\par
	\takto{}{\e\z\g\z\z\z}{\x}\par
	\takto{}{\f\z\a\z\z\z}{\x}\par
	\takto{}{\e\z\z\z\B\z}{\x}\par
	\takto{}{\c\z\z\z\z\z}{\x\x}\newpage
	\takto{}{\e\f\g\z\z\z}{\x}\par
	\takto{}{\f\g\a\z\z\z}{\x}\par
	\takto{}{\e\f\g\z\z\z}{\x}\par
	\takto{}{\d\e\f\z\z\z}{\x}\par
	\takto{}{\e\f\g\z\z\z}{\x}\par
	\takto{}{\f\g\a\z\z\z}{\x}\par
	\takto{}{\e\z\d\c\B\z}{\x}\par
	\takto{}{\c\z\z\z\z\z}{\X}
\end{tabulaturo}

%%\muzikajxo[
titolo-he     = {חצות־הלילה של מר דוולנד},
titolo-eo     = {La noktomezo de S-ro Dowland},
titolo-xx     = {Mr Dowland’s Midnight},
ikono         = {🕛},
komponisto-he = {ג׳ון דוולנד},
komponisto-eo = {John Dowland},
komponisto-xx = {},
noto-he       = {ג׳ון דוולנד (1563–1626) היה מלחין ונגן לאוטה מתקופת הרנסנס. את מרבית המוסיקה שכתב הוא כתב ללאוטה, סולו או בליווי קולי או כלי; ככלל גם המוסיקה וגם המילים מלנכוליות (אחת היצירות המפורסמות שלו נקראת \Lat{Semper Dowland, semper dolens} „תמיד דוולנד, תמיד דואב”…). כאן עיבדתי את הקול הראשון של „חצות־הלילה של מר דוולנד”, מתוך הטבלטורה שבספר הלאוטה של מרגרט בורד.\pnurletikedo{gerbode.net/sources/spencer_private_library/margaret_board_lute_book/pdf/086_midnight_dowland.pdf}{boardmidnight}.},
noto-eo       = {John Dowland estis compnisto kaj ludisto de liuto de la Renesanco. La plejparto de gia verkaro estas skribita por liuto, kaj sole kaj kun voĉa aŭ instrumenta akompano; ĝenerale, kaj la muziko kaj la kantoteksto estas melanĥoliaj (unu le gia plej famaj muzikaĵoj estas nomata «Semper Dowland, semper dolens» (Ĉiam Dowland, ĉiam malĝoja)…). Tieĉi mi aranĝis la unua voĉo de «La noktomezo de S-ro Dowland» de la tabulaturo el la Libro de Liuto de Margaret Board.\pnurlref{gorkogo}},
]{midnight}

\begin{tabulaturo}
	\takto{}{\d\z\z\e\f\z\d\z}{}\par
	\takto{}{\f\z\g\z\e\z\cis\z}{\x}\par
	\takto{}{\d\z\z\e\f\z\e\z}{}\par
	\takto{}{\f\z\g\z\a\z\z\z}{\x}\par
	\takto{}{\d\z\d\e\f\z\d\z}{}\par
	\takto{}{\f\z\g\z\e\z\cis\z}{\x}\par
	\takto{}{\d\z\d\e\f\z\e\z}{}\par
	\takto{}{\d\e\f\g\a\z\z\z}{\r{0}{1}}\newpage
	\takto{}{\e\z\C\z\g\z\a\z}{}\par
	\takto{}{\bes\z\a\g\f\z\g\z}{\x}\par
	\takto{}{\a\z\g\a\g\z\f\z}{}\par
	\takto{}{\e\z\e\z\d\z\a\bes}{\x}\par
	\takto{}{\C\z\bes\a\g\f\g\a}{}\par
	\takto{}{\bes\z\a\g\f\e\f\g}{\x}\par
	\takto{}{\a\z\g\a\bes\a\g\f}{}\par
	\takto{}{\e\d\e\z\d\z\z\z}{\R{1}{0}}\par
\end{tabulaturo}

%%\muzikajxo[
titolo-he     = {טוטורו},
titolo-eo     = {Totoro},
titolo-xx     = {\Jap{となりのトトロ}},
ikono         = {🌳},
komponisto-he = {ג׳ו היסאישי},
komponisto-eo = {Ĝo Hisaiŝi},
komponisto-xx = {\Jap{久石譲}},
noto-he       = {},
noto-eo       = {},
]{totoro}

\begin{tabulaturo}
	\takto{}{\f\z\e\z\f\c\z\z}{\x}\par
	\takto{}{\z\z\z\z\z\z\z\z}{\x}\par
	\takto{}{\f\z\e\z\f\a\z\z}{\x}\par
	\takto{}{\z\z\z\z\z\z\bes\z}{\x}\par
	\takto{}{\a\z\g\z\f\bes\z\z}{\x}\par
	\takto{}{\a\z\g\z\f\z\f\z}{\x}\par
	\takto{}{\z\g\g\z\z\z\z\w}{\x}\par
	\takto{}{\f\z\e\z\f\c\z\z}{\x}\par
	\takto{}{\z\z\z\z\z\z\w\w}{\x}\par
	\takto{}{\f\z\e\z\f\C\z\z}{\x}\newpage
	\takto{}{\z\z\z\z\w\w\w\bes}{\x}\par
	\takto{}{\bes\bes\bes\z\a\g\bes\z}{\x}\par
	\takto{}{\z\z\z\g\a\bes\a\z}{\x}\par
	\takto{}{\a\z\a\g\f\a\z\z}{\x}\par
	\takto{}{\z\z\z\z\w\w\d\z}{\x}\par
	\takto{}{\e\z\f\z\g\d\z\d}{\x}\par
	\takto{}{\e\z\f\g\f\C\z\z}{\x}\par
	\takto{}{\z\z\z\z\z\z\z\z}{\x}\newpage
	\takto{}{\z\w\f\g\a\bes\C\z}{\x}\par
	\takto{}{\a\f\z\C\z\bes\z\g}{\x}\par
	\takto{}{\z\z\w\w\bes\z\g\e}{\x}\par
	\takto{}{\z\bes\z\a\z\f\z\z}{\x}\par
	\takto{}{\w\w\cis\z\f\z\bes\z}{\x}\par
	\takto{}{\a\z\C\f\z\w\w\a}{\x}\par
	\takto{}{\bes\a\bes\a\bes\a\f\g}{\x}\par
	\takto{}{\z\z\w\w\f\g\a\bes}{\x}\par
	\takto{}{\C\z\a\f\z\C\z\bes}{\x}\par
	\takto{}{\z\g\z\z\w\w\bes\z}{\x}\newpage
	\takto{}{\g\e\z\bes\z\a\z\f}{\x}\par
	\takto{}{\z\z\w\w\d\z\d\z}{\x}\par
	\takto{}{\C\bes\a\bes\C\z\z\f}{\x}\par
	\takto{}{\f\z\w\f\bes\a\f\bes}{\x}\par
	\takto{}{\a\f\d\C\z\z\z\z}{\x}\par
	\takto{}{\z\z\z\z\w\w\c\c}{\x}\par
	\takto{}{\bes\a\g\a\f\z\z\z}{\x}\par
	\takto{}{\z\z\z\z\z\z\z\z}{\X}\par
\end{tabulaturo}
 % ⚙
%%\muzikajxo[
titolo-he     = {השכן שלי טוטורו: סמפו},
titolo-eo     = {Mia Najbaro Totoro: Sampo},
titolo-xx     = {\Jap{となりのトトロ:さんぽ}},
ikono         = {🐛},
komponisto-he = {ג׳ו היסאישי},
komponisto-eo = {Ĝo Hisaiŝi},
komponisto-xx = {\Jap{久石譲}},
noto-he       = {נעימת הפתיחה של הסרט, שמנוגנת כשרצות הכתוביות},
noto-eo       = {La melodio de la komenco de la filmo, ludita dum la komencaj titoloj aperas.},
]{totoro}

\begin{tabulaturo}
	\takto{}{\e\z\z\z\g\z\z\z}{}\par
	\takto{}{\C\z\z\z\z\z\z\z}{\x}\par
	\takto{}{\g\z\z\z\a\z\z\z}{}\par
	\takto{}{\g\z\z\z\z\z\z\z}{\x}\par
	\takto{}{\w\w\w\c\e\z\z\g}{}\par
	\takto{}{\C\z\z\z\b\z\z\a}{\x}\par
	\takto{}{\g\z\z\z\z\z\z\z}{}\par
	\takto{}{\z\z\z\z\w\w\w\w}{\x}\newpage
	\takto{}{\a\z\z\a\a\z\z\a}{}\par
	\takto{}{\z\z\z\C\b\z\z\a}{\x}\par
	\takto{}{\g\z\z\z\z\z\z\z}{}\par
	\takto{}{\z\z\z\z\w\w\w\w}{\x}\par
	\takto{}{\a\z\z\g\a\z\z\g}{}\par
	\takto{}{\d\z\z\z\e\z\z\z}{\x}\par
	\takto{}{\c\z\z\z\z\z\z\z}{}\par
	\takto{}{\z\z\z\z\w\w\w\w}{\x}\newpage
	\takto{}{\aes\z\z\aes\aes\z\z\aes}{}\par
	\takto{}{\z\z\z\z\z\z\z\z}{\x}\par
	\takto{}{\g\z\z\g\g\z\z\g}{}\par
	\takto{}{\z\z\z\z\z\z\z\z}{\x}\par
	\takto{}{\f\z\z\z\f\z\z\z}{}\par
	\takto{}{\f\z\z\z\d\z\z\e}{\x}\par
	\takto{}{\z\z\z\z\z\z\z\z}{}\par
	\takto{}{\z\z\z\z\w\w\w\w}{\x}\newpage
	\takto{}{\c\z\z\z\C\z\z\C}{}\par
	\takto{}{\b\z\z\z\g\z\z\a}{\x}\par
	\takto{}{\z\z\z\z\z\z\z\z}{}\par
	\takto{}{\w\w\w\a\b\z\z\C}{\x}\par
	\takto{}{\d\z\z\z\c\z\z\z}{}\par
	\takto{}{\b\z\z\z\a\z\z\z}{\x}\par
	\takto{}{\g\z\z\z\z\z\z\z}{}\par
	\takto{}{\z\z\z\z\w\w\w\w}{\r{0}{1}}\newpage
	\takto{}{\C\z\z\b\C\z\z\g}{}\par
	\takto{}{\e\z\z\C\z\z\z\b}{\x}\par
	\takto{}{\z\z\z\z\z\z\z\z}{}\par
	\takto{}{\z\z\z\z\g\z\z\g}{\x}\par
	\takto{}{\a\z\z\z\w\w\w\w}{}\par
	\takto{}{\b\z\z\z\w\w\w\w}{\x}\par
	\takto{1}{\C\z\z\z\z\z\z\z}{}\par
	\takto{}{\z\z\z\z\w\w\w\w}{\r{1}{0}}\par
	\takto{2}{\C\z\z\w\C\C\C\w}{}\par
	\takto{}{\C\z\z\z\z\z\z\z}{\X}{}
\end{tabulaturo}
 % ⚙
%%%⚙ https://en.wikipedia.org/wiki/Tourdion

\muzikajxo[
titolo-he     = {טורדיון},
titolo-eo     = {Tordiono},
titolo-xx     = {Tordion},
ikono         = {🔀},
komponisto-he = {אלמוני/ת},
komponisto-eo = {\emph{sennoma}},
komponisto-xx = {},
noto-he       = {טורדיון (מ־\Lat{tordre} בצרפתית, „לפתל”) הוא ריקוד, במקצב משולש, שהיה נפוץ בחלקים של אירופה בתקופת הרנסנס; דומה לריקוד גליארד אבל מהיר וחלק ממנו. הטורדיון המפורסם ביותר כיום הוא זה שעובד כאן; המלחין לא ידוע, והתווים הודפסו בשנת 1530 על־ידי המדפיס פייר אטניין\pnurl{he.wikipedia.org/wiki/פייר_אטניין} כחלק מ„לה מגדלנה”.},
noto-eo       = {Tordiono (el «tordre» (tordi) en la franca lingvo) estas tritakta danco, kiu estis disvastiĝinta en partoj de Eŭropo en la Renesanco; ĝi estas simila al la galjardo, sed estas pli rapida kaj pli glata. Nuntempe, la plej fama tordiono estas tiu, kiu estas aranĝita tie ĉi; la komponisto estas nekonata, kaj la muzikaĵo estis presita en 1530 per la presisto Pierre Attaignant\pnurl{eo.wikipedia.org/wiki/Pierre_Attaignant}, kiel parto de «La Magdalena».},
]{tordion}

\begin{tabulaturo}
	\takto{}{\d\e\f\g\f\e}{\x}\par
	\takto{}{\d\z\z\e\f\g}{\x}\par
	\takto{}{\a\g\f\f\g\e}{\x}\par
	\takto{}{\f\z\e\d\c\z}{\x}\par
	\takto{}{\d\e\f\g\f\e}{\x}\par
	\takto{}{\d\z\f\z\e\z}{\x}\par
	\takto{}{\d\z\z\z\c\z}{\x}\par
	\takto{}{\d\z\z\z\z\z}{\r{1}{1}}\newpage
	\takto{}{\a\z\z\g\a\bes}{\x}\par
	\takto{}{\a\z\z\z\a\z}{\x}\par
	\takto{}{\C\bes\a\g\f\e}{\x}\par
	\takto{}{\f\z\z\e\d\z}{\x}\par
	\takto{}{\a\z\z\g\a\bes}{\x}\par
	\takto{}{\a\z\g\f\e\z}{\x}\par
	\takto{}{\d\z\z\z\c\z}{\x}\par
	\takto{}{\d\z\z\z\z\z}{\R{1}{0}}
\end{tabulaturo}

%%\muzikajxo[
titolo-he     = {יום־הולדת},
titolo-eo     = {Naskiĝtago},
titolo-xx     = {},
ikono         = {🎈},
komponisto-he = {עממי},
komponisto-eo = {\emph{popola}},
komponisto-xx = {},
noto-he       = {},
noto-eo       = {},
]{taxtaxtax}

\begin{tabulaturo}
	\takto{}{\d\z\d\z\a\z\g\z}{\x}\par
	\takto{}{\a\z\g\z\f\e\d\z}{\x}\par
	\takto{1}{\d\z\d\z\a\z\z\g}{\x}\par
	\takto{}{\f\z\g\z\a\z\z\z}{\r{1}{0}}\par
	\takto{2}{\a\z\a\bes\a\z\z\g}{\x}\par
	\takto{}{\f\z\e\z\d\z\z\z}{\x}\par
	\takto{}{\bes\z\z\a\bes\z\z\a}{\x}\newpage
	\takto{}{\g\z\g\z\g\z\z\z}{\x}\par
	\takto{}{\a\z\z\g\a\z\z\g}{\x}\par
	\takto{}{\f\z\f\z\f\z\z\z}{\r{0}{1}}\par
	\takto{}{\d\e\f\z\e\f\g\z}{\x}\par
	\takto{}{\f\g\a\z\f\e\d\z}{\x}\par
	\takto{}{\d\e\f\z\e\f\g\z}{\x}\par
	\takto{}{\a\z\a\a\d\z\z\z}{\r{1}{0}}
\end{tabulaturo}

%%\muzikajxo[
titolo-he     = {יום הולדת שמח},
titolo-eo     = {Feliĉan naskiĝtagon},
titolo-xx     = {Happy Birthday to You},
ikono         = {🎈},
komponisto-he = {?},
komponisto-eo = {?},
komponisto-xx = {},
noto-he       = {לפי ספר השיאים של גינס, השיר הזה הוא השיר הידוע ביותר בשפה האנגלית. הזהות של מי שהלחינו את המנגינה, או חיברו את המילים (הטריוויאליות), לא ברורה באופן מוחלט, וכנראה תשאר בערפילי ההיסטוריה.},
noto-eo       = {Laŭ la Guinness-libro de rekordoj, tiu ĉi kanto estas la plej konata anglalingva kanto. La idento de la komponisto kaj la verkisto de la kantoteksto (kiu sincere estas… banala) ne estas definitiva.},
]{birthday}

\begin{tabulaturo}
	\takto{}{\c\c}{\x}\par
	\takto{}{\d\z\c\z\f\z}{\x}\par
	\takto{}{\e\z\z\z\c\c}{\x}\par
	\takto{}{\d\z\c\z\g\z}{\x}\par
	\takto{}{\f\z\z\z\c\c}{\x}\par
	\takto{}{\C\z\a\z\f\z}{\x}\par
	\takto{}{\e\z\d\z\ais\ais}{\x}\par
	\takto{}{\a\z\f\z\g\z}{\x}\par
	\takto{}{\f\z\z\z\z\z}{\X}
\end{tabulaturo}

%%\muzikajxo[
titolo-he     = {יונתן הקטן},
titolo-eo     = {Hanseto la Malgranda},
titolo-xx     = {Hänschen klein},
ikono         = {🌳},
komponisto-he = {עממי},
komponisto-eo = {\emph{popola}},
komponisto-xx = {},
noto-he       = {},
noto-eo       = {},
]{jonatan}

\begin{tabulaturo}
	\takto{}{\g\e\e\z}{\x}
	\takto{}{\f\d\d\z}{\x}\par
	\takto{}{\c\d\e\f}{\x}
	\takto{}{\g\g\g\z}{\x}\par
	\takto{}{\g\e\e\z}{\x}
	\takto{}{\f\d\d\z}{\x}\par
	\takto{}{\c\e\g\g}{\x}
	\takto{}{\c\z\z\z}{\x}\par
	\takto{}{\d\d\d\d}{\x}
	\takto{}{\d\e\f\z}{\x}\par
	\takto{}{\e\e\e\e}{\x}
	\takto{}{\e\f\g\z}{\x}\par
	\takto{}{\g\e\e\z}{\x}
	\takto{}{\f\d\d\z}{\x}\par
	\takto{}{\c\e\g\g}{\x}
	\takto{}{\c\z\z\z}{\X}
\end{tabulaturo}

%%\muzikajxo[
titolo-he     = {לה פוליה},
titolo-eo     = {La Folia},
titolo-xx     = {},
ikono         = {ℱ},
komponisto-he = {אלמוני/ת},
komponisto-eo = {\emph{sennoma}},
komponisto-xx = {},
noto-he       = {„לה פוליה” היא נושא מוזיקלי שזכה להיות משובץ ביצירותיהם של מלחינים רבים; חלקם כתבו לו וריאציות (כלומר, חזרות על אותו נושא בסיסי, כל פעם בלבוש שונה) וחלקם שלבו אותו ביצירות אחרות. הגרסה כאן מבוססת על הנושא הבסיסי שמוצג לפני 23 הווריאציות\pnurletikedo{imslp.org/wiki/12_Violin_Sonatas,_Op.5_(Corelli,_Arcangelo)}{corellifollia} אצל קורלי\pnurl{he.wikipedia.org/wiki/ארכאנג'לו_קורלי}.},
noto-eo       = {«La Folia» estas muzika temo, kiun multaj komponsitoj uzis en siaj verkoj, kiel variaĵoj (t.~e.\ ripetado de la sama baza temo, ĉiun fojon malsame) aŭ enteksita en aliaj verkoj. La versio, kiu estas ĉitie, estas la temo, kiu antaŭas la 23 variaĵoj\pnurlref{corellifollia} je Corelli\pnurl{eo.wikipedia.org/wiki/Arcangelo_Corelli}}.
]{follia}

\begin{tabulaturo}
	\takto{}{\d\z\d\z\z\e}{\x}\par
	\takto{}{\cis\z\z\z\cis\z}{\x}\par
	\takto{}{\d\z\d\z\z\z}{\x}\par
	\takto{}{\e\z\z\z\e\z}{\x}\par
	\takto{}{\f\z\f\z\z\g}{\x}\par
	\takto{1}{\e\z\z\z\e\z}{\x}\par
	\takto{}{\d\cis\d\z\z\e}{\x}\par
	\takto{}{\cis\z\z\z\cis\z}{\r{1}{0}}\par
	\takto{2}{\e\z\z\e\f\z}{\x}\par
	\takto{}{\d\z\d\z\z\cis}{\x}\par
	\takto{}{\d\z\z\z\z\z}{\X}\par
\end{tabulaturo}

%%\muzikajxo[
titolo-he     = {לו ספניולטו},
titolo-eo     = {La hispaneto},
titolo-xx     = {Lo spagnoletto},
ikono         = {\Lat{Ñ}},
komponisto-he = {אלמוני/ת},
komponisto-eo = {\emph{sennoma}},
komponisto-xx = {},
noto-he       = {הגרסה כאן מבוססת על אוסף המנגינות לריקודים שקיבץ צ׳זרה נגרי (1535 לערך~– 1605 לערך), כטבלטורה ללאוטה.\pnurletikedo{gerbode.net/sources/negri_nuove_inventione_1604/pdf/01_lo_spagnoletto.pdf}{spagnolettoliuto}},
noto-eo       = {Nia koncerna versio baziĝas sur la dancmuzikaĵaro far Cesare Negri (ĉ.~1535~– ĉ.~1605), kiel liutotabulaturo.\pnurlref{spagnolettoliuto}},
]{spagnoletto}

\begin{tabulaturo}
	\takto{}{\e\z}{\r{0}{1}}\par
	\takto{}{\g\z\fis\g\a\z\g\a}{\x}\par
	\takto{}{\b\z\b\z\g\z\b\z}{\x}\par
	\takto{}{\a\z\g\z\a\z\fis\z}{\x}\par
	\takto{1}{\g\z\z\z\z\z\e\z}{\r{1}{0}}\par
	\takto{2}{\g\z\z\z\z\z\b\z}{\r{0}{1}}\newpage
	\takto{}{\a\z\g\z\fis\z\e\z}{\x}\par
	\takto{}{\dis\z\z\z\z\z\b\a}{\x}\par
	\takto{}{\g\z\fis\e\dis\z\fis\z}{\x}\par
	\takto{1}{\e\z\z\z\z\z\b\z}{\r{1}{0}}\par
	\takto{2}{\e\z\z\z\d\z\e\z}{\r{0}{1}}\par
	\takto{}{\fis\z\z\z\d\z\fis\g}{\x}\par
	\takto{}{\a\z\z\z\fis\z\b\a}{\x}\par
	\takto{}{\g\z\fis\e\dis\z\fis\z}{\x}\par
	\takto{1}{\e\z\z\z\d\z\e\z}{\r{1}{0}}\par
	\takto{2}{\e\z\z\z\z\z\z\z}{\X}\par
\end{tabulaturo}

%%\muzikajxo[
titolo-he     = {לכובע שלי},
titolo-eo     = {Mia ĉapelo},
titolo-xx     = {},
ikono         = {△},
komponisto-he = {עממי},
komponisto-eo = {\emph{popola}},
komponisto-xx = {},
noto-he       = {לא ידוע מאת מי המילים או הלחן.\pnurl{zemer.co.il/song.asp?id=1934}},
noto-eo       = {La verkistoj kaj de la kantoteksto kaj de la melodio estas nekonataj.},
]{kova}

\begin{tabulaturo}
	\takto{}{\c}{\x}\par
	\takto{}{\f\f\g}{\x}
	\takto{}{\a\z\c}{\x}\par
	\takto{}{\f\z\a}{\x}
	\takto{}{\g\z\c}{\x}\par
	\takto{}{\g\z\a}{\x}
	\takto{}{\bes\g\e}{\x}\par
	\takto{}{\c\d\e}{\x}
	\takto{}{\f\z\c}{\x}\par
	\takto{}{\f\z\g}{\x}
	\takto{}{\a\a\c}{\x}\par
	\takto{}{\f\z\a}{\x}
	\takto{}{\g\z\z}{\x}\par
	\takto{}{\g\z\a}{\x}
	\takto{}{\bes\g\e}{\x}\par
	\takto{}{\c\d\e}{\x}
	\takto{}{\f\z\z}{\X}
\end{tabulaturo}

%%\muzikajxo[
titolo-he     = {לשדה יצוא יצאתי},
titolo-eo     = {Barieroj},
titolo-xx     = {{\hebrewfont באַריקאַדן}},
ikono         = {🐎},
komponisto-he = {עממי},
komponisto-eo = {\emph{popola}},
komponisto-xx = {},
noto-he       = {מקור הלחן לא ידוע, כמו גם מקור המילים העבריות. לאותו הלחן קיים גם שיר יידי, באַריקאַדן „בריקדות”\pnurletikedo{yidlid.org/chansons/barikadn/}{barikadn}, פרי עטו של שמערקע קאַטשערגינסקי\pnurl{he.wikipedia.org/wiki/שמריהו_קצ'רגינסקי} (1908–1954), המתאר את התקוממות פועלים בלודז׳; ביצוע של השיר מ־1948 בידי קאַטשערגינסקי ורות רובין\pnurletikedo{ololo.fm/search/Shmerke+Kacherginsky+\%28And+Ruth+Rubin\%29/Barikadn+1948}{barikadn1948}, וביצוע מודרני בידי הלהקה \Lat{The Klezmatics}\pnurletikedo{youtu.be/h7FZ-hvPRx8}{barikadnklezmatics}.},
noto-eo       = {La origino de la melodio aŭ la hebrea kantoteksto estas nekonata. La sama melodio havas ankaŭ, jida, kantoteksto, \emph{Barikadn} «Barieroj», far Ŝmerke Kaĉerginski (1908–1954), kiu priskribas la laborista popolleviĝo en Lodzo; kantita per Kaĉerginski kaj Rut Rubin\pnurlref{barikadn1948}, kaj per la muzikgrupo \emph{The Klezmatics}\pnurlref{barikadnklezmatics}.}
]{sade}

\begin{tabulaturo}
	\takto{}{\d\a\a\a\a\a\a\a}{\x}\par
	\takto{}{\a\g\g\f\a\z\a\w}{\x}\par
	\takto{}{\d\a\a\a\a\a\a\a}{\x}\par
	\takto{}{\a\g\g\f\f\z\f\w}{\x}\par
	\takto{}{\d\g\d\g\f\f\f\w}{\x}\par
	\takto{}{\f\e\e\d\f\z\f\w}{\x}\par
	\takto{}{\d\g\d\g\f\f\f\w}{\x}\par
	\takto{}{\f\e\e\d\d\z\d\z}{\X}
\end{tabulaturo}

%%\muzikajxo[
titolo-he     = {מוקדם בבוקר},
titolo-eo     = {Frumatene},
titolo-xx     = {Early One Morning},
ikono         = {🌅},
komponisto-he = {עממי},
komponisto-eo = {\emph{popola}},
komponisto-xx = {},
noto-he       = {שיר עם באנגלית מהאי בריטניה (מס׳ ראוד\pnurletikedo{en.wikipedia.org/wiki/Roud_Folk_Song_Index}{roud} 12682).},
noto-eo       = {Popola anglalingva kanto el la insulo Britio (Roud\pnurlref{roud} 12682)},
]{early}

\begin{tabulaturo}
	\takto{}{\c\z\c\c}{\x}
	\takto{}{\c\e\g\g}{\x}\par
	\takto{}{\a\f\d\c}{\x}
	\takto{}{\B\d\B\B}{\x}\par
	\takto{}{\c\z\c\c}{\x}
	\takto{}{\c\e\g\g}{\x}\par
	\takto{}{\a\f\d\B}{\x}
	\takto{}{\c\z\z\z}{\r{0}{1}}\par
	\takto{}{\d\z\e\f}{\x}
	\takto{}{\g\e\c\z}{\x}\par
	\takto{}{\d\z\e\f}{\x}
	\takto{}{\g\e\c\z}{\x}\par
	\takto{}{\c\e\g\C}{\x}
	\takto{}{\b\a\g\f}{\x}\par
	\takto{}{\e\d\c\B}{\x}
	\takto{}{\c\z\z\z}{\R{1}{0}}
\end{tabulaturo}

%%\muzikajxo[
titolo-he     = {סקורה סקורה},
titolo-eo     = {Sakura Sakura},
titolo-xx     = {\Jap{さくらさくら}},
ikono         = {🌸},
komponisto-he = {עממי},
komponisto-eo = {Popola},
komponisto-xx = {},
]{sakura}

\begin{tabulaturo}
	\takto{}{\a\z\a\z\b\z\z\z}{\x}\par
	\takto{}{\a\z\a\z\b\z\z\z}{\x}\par
	\takto{}{\a\z\b\z\C\z\b\z}{\x}\par
	\takto{}{\a\z\b\a\f\z\z\z}{\x}\par
	\takto{}{\e\z\c\z\e\z\f\z}{\x}\par
	\takto{}{\e\z\e\c\B\z\z\z}{\x}\par
	\takto{}{\a\z\b\z\C\z\b\z}{\x}\newpage
	\takto{}{\a\z\b\a\f\z\z\z}{\x}\par
	\takto{}{\e\z\c\z\e\z\f\z}{\x}\par
	\takto{}{\e\z\e\c\B\z\z\z}{\x}\par
	\takto{}{\a\z\a\z\b\z\z\z}{\x}\par
	\takto{}{\a\z\a\z\b\z\z\z}{\x}\par
	\takto{}{\e\z\f\z\b\a\f\z}{\x}\par
	\takto{}{\e\z\z\z\z\z\z\z}{\x}
\end{tabulaturo}

%%\muzikajxo[
titolo-he     = {קנטיגה~353},
titolo-eo     = {Kantigo~353},
titolo-xx     = {},
ikono         = {♍},
komponisto-he = {עממי~/ אלפונזו העשירי},
komponisto-eo = {Popola / Alfonso la 10-a},
komponisto-xx = {},
]{353}

\begin{tabulaturo}
	\takto{}{\d\z\a\z\a\z\z\g}{\x}\par
	\takto{}{\f\z\f\g\a\z\z\d}{\x}\par
	\takto{}{\d\z\a\z\a\z\z\g}{\x}\par
	\takto{}{\f\e\d\z\f\z\z\z}{\x}\par
	\takto{}{\d\z\a\z\a\z\z\g}{\x}\par
	\takto{}{\f\z\f\g\a\z\z\d}{\x}\par
	\takto{}{\d\z\a\z\a\z\z\g}{\x}\par
	\takto{}{\f\e\d\c\d\z\z\z}{\x}\newpage
	\takto{}{\a\z\b\z\g\z\z\a}{\x}\par
	\takto{}{\b\z\C\b\a\z\z\g}{\x}\par
	\takto{}{\a\z\a\b\a\z\z\g}{\x}\par
	\takto{}{\f\z\g\z\f\z\z\z}{\x}\par
	\takto{}{\a\z\b\z\g\z\z\a}{\x}\par
	\takto{}{\b\z\C\b\a\z\z\g}{\x}\par
	\takto{}{\a\z\a\b\a\z\z\g}{\x}\par
	\takto{}{\f\e\d\c\d\z\z\z}{\X}\par
\end{tabulaturo}

%%\muzikajxo[
titolo-he     = {ריקוד הציפורים},
titolo-eo     = {Birdeta danco},
titolo-xx     = {Der Ententanz},
ikono         = {🐦},
komponisto-he = {ורנר תומס},
komponisto-eo = {Werner Thomas},
komponisto-xx = {},
noto-he       = {במקור הריקוד נקרא \Lat{Der Ententanz} „ריקוד הברווז”, בגרמנית. המלחין, ורנר תומאס, שהיה נגן אקורדיון, החל לנגן את המנגינה בשנת 1963 במסעדות ובמלונות שונים. באחת מהופעותיו, שמע את המנגינה מפיק בשם לואי ואן ריימנאנט. בשלב מאוחר יותר ואן ריימנאנט כתב לשיר מילים ובשנת 1970 החל בהפצתו לציבור, כמעט ללא הצלחה. בשנת 1977 להקה בשם \Lat{De Electronica} יצרה גרסה אינסטרומנטלית לשיר, שהפכה ללהיט עולמי מצליח מאוד.},
noto-eo       = {La origina germana nomo de la danco estas \emph{Der Ententanz} «La danco de la anaso». La komponisto, Werner Thomas, kiu estas akordionisto, komencis ludi la melodion ekde 1963 en restoracioj kaj hoteloj. Unufoje la produktoro Louis van Rijmenant aŭdis la melodion; gi verkis vortojn por la melodio, kaj en 1970 komencis publikigi ĝin, sensukcese. En 1977 la muzikgrupo De Electronica kreis instrumentan version, kiu ekestis tutmonda furoraĵeto.},
]{chipchip}

\begin{tabulaturo}
	\takto{}{\g\g\a\a\e\e}{\x}\par
	\takto{}{\g\z\g\g\a\a\e\e}{\x}\par
	\takto{}{\g\z\g\g\a\a\C\C}{\x}\par
	\takto{}{\b\z\b\z\a\z\g\z}{\x}\par
	\takto{}{\f\z\f\f\g\g\d\d}{\x}\par
	\takto{}{\f\z\f\f\g\g\d\d}{\x}\par
	\takto{1}{\f\z\f\f\g\g\b\b}{\x}\par
	\takto{}{\a\z\a\z\g\z\f\z}{\x}\par
	\takto{}{\e\z}{\r{1}{0}}\par
	\takto{2}{\f\z\g\g\a\a\b\b}{\x}\par
	\takto{}{\C\z\z\z\z\z\z\z}{\X}
\end{tabulaturo}
 % ⚙
%%\muzikajxo[
titolo-he     = {שאגיד לך, אמא?},
titolo-eo     = {Ha! ĉu mi diros al vi, panjo?},
titolo-xx     = {Ah~! vous dirai-je, maman},
ikono         = {🔤},
komponisto-he = {עממי},
komponisto-eo = {\emph{popola}},
komponisto-xx = {},
noto-he       = {},
noto-eo       = {},
]{maman}

\begin{tabulaturo}
	\takto{}{\c\c\g\g}{\x}
	\takto{}{\a\a\g\z}{\x}\par
	\takto{}{\f\f\e\e}{\x}
	\takto{}{\d\d\c\z}{\x}\par
	\takto{}{\g\g\f\f}{\x}
	\takto{}{\e\e\d\z}{\x}\par
	\takto{}{\g\g\f\f}{\x}
	\takto{}{\e\e\d\z}{\x}\par
	\takto{}{\c\c\g\g}{\x}
	\takto{}{\a\a\g\z}{\x}\par
	\takto{}{\f\f\e\e}{\x}
	\takto{}{\d\d\c\z}{\X}
\end{tabulaturo}
 % ⚙
%%\muzikajxo[
titolo-he     = {שיר המסיכות},
titolo-eo     = {La kanto de la maskoj},
titolo-xx     = {},
ikono         = {👹},
komponisto-he = {נחום נרדי},
komponisto-eo = {Naĥum Nardi},
komponisto-xx = {},
noto-he       = {},
noto-eo       = {},
]{zakan}
%⚙ http://zemer.co.il/song.asp?id=232

\begin{tabulaturo}
	\takto{}{\c}{\x}\par
	\takto{}{\f\e\d\c\d\e}{\x}\par
	\takto{}{\f\z\c\z\w\c}{\x}\par
	\takto{}{\f\e\d\c\d\e}{\x}\par
	\takto{}{\f\z\c\z\w\c}{\r{0}{1}}\par
	\takto{}{\f\z\f\g\f\g}{\x}\par
	\takto{}{\a\z\C\a\z\f}{\x}\par
	\takto{}{\g\z\g\g\a\g}{\x}\par
	\takto{1}{\f\w\C\C\w\c}{\r{1}{0}}\par
	\takto{2}{\f\z\z\z\z\z}{\X}
\end{tabulaturo}
 % ⚙
%%\muzikajxo[
titolo-he     = {שיר סיראי הוולגה},
titolo-eo     = {La kanto de la boatistoj de la Volga},
titolo-xx     = {\Rus{Эй, ухнем!}},
ikono         = {⛵},
komponisto-he = {עממי},
komponisto-eo = {Popola},
komponisto-xx = {},
]{volga}

\begin{tabulaturo}
	\takto{}{\g\z\e\z\a\z\z\z}{\x}\par
	\takto{}{\e\z\z\z\z\z\fis\z}{\x}\par
	\takto{}{\g\z\e\z\a\z\z\z}{\x}\par
	\takto{}{\e\z\z\z\z\z\fis\z}{\x}\par
	\takto{}{\g\z\z\z\C\z\z\z}{\x}\par
	\takto{}{\b\z\C\b\a\z\z\z}{\x}\par
	\takto{}{\g\z\e\z\a\z\z\z}{\x}\par
	\takto{}{\e\z\z\z\z\z\z\z}{\R{0}{1}}\newpage
	\takto{}{\g\z\z\z\z\z\g\z}{\x}\par
	\takto{}{\g\z\f\z\e\z\d\z}{\x}\par
	\takto{}{\c\z\z\z\g\z\z\z}{\x}\par
	\takto{}{\e\z\z\z\z\z\z\z}{\r{1}{0}}\par
	\takto{}{\a\z\z\z\a\z\a\z}{\x}\par
	\takto{}{\e\z\z\z\e\z\z\z}{\x}\par
	\takto{}{\C\z\z\z\b\z\a\z}{\x}\par
	\takto{}{\g\z\z\z\e\z\z\z}{\x}\par
	\takto{}{\a\z\z\z\C\z\z\z}{\x}\par
	\takto{}{\b\z\C\b\a\z\z\z}{\x\reludu}
\end{tabulaturo}
 % ⚙
\muzikajxo[
titolo-he     = {שרוולים ירוקים},
titolo-eo     = {Verdaj Manikoj},
titolo-xx     = {Greensleeves},
ikono         = {👕},
komponisto-he = {עממי},
komponisto-eo = {Popola},
komponisto-xx = {},
noto-he       = {},
noto-eo       = {},
]{greensleeves}

\begin{tabulaturo}
	\takto{}{\d\z}{\x}\par
	\takto{}{\f\z\z\z\g\z}{}\par
	\takto{}{\a\z\z\b\a\z}{\x}\par
	\takto{}{\g\z\z\z\e\z}{}\par
	\takto{}{\c\z\z\d\e\z}{\x}\par
	\takto{}{\f\z\z\z\d\z}{}\par
	\takto{}{\d\z\z\des\d\z}{\x}\par
	\takto{}{\e\z\z\z\f\z}{}\par
	\takto{}{\e\z\z\z\d\z}{\x}\newpage
	\takto{}{\f\z\z\z\g\z}{}\par
	\takto{}{\a\z\z\b\a\z}{\x}\par
	\takto{}{\g\z\z\z\e\z}{}\par
	\takto{}{\c\z\z\d\e\z}{\x}\par
	\takto{}{\f\z\z\e\d\z}{}\par
	\takto{}{\des\z\z\d\e\z}{\x}\par
	\takto{}{\d\z\z\z\z\z}{}\par
	\takto{}{\z\z\z\z\z\z}{\x}\newpage
	\takto{}{\C\z\z\z\z\z}{}\par
	\takto{}{\C\z\z\b\a\z}{\x}\par
	\takto{}{\g\z\z\z\e\z}{}\par
	\takto{}{\c\z\z\d\e\z}{\x}\par
	\takto{}{\f\z\z\z\d\z}{}\par
	\takto{}{\d\z\z\des\d\z}{\x}\par
	\takto{}{\e\z\z\z\f\z}{}\par
	\takto{}{\e\z\z\z\z\z}{\x}\newpage
	\takto{}{\C\z\z\z\z\z}{}\par
	\takto{}{\C\z\z\b\a\z}{\x}\par
	\takto{}{\g\z\z\z\e\z}{}\par
	\takto{}{\c\z\z\d\e\z}{\x}\par
	\takto{}{\f\z\z\e\d\z}{}\par
	\takto{}{\des\z\z\d\e\z}{\x}\par
	\takto{}{\d\z\z\z\z\z}{}\par
	\takto{}{\z\z\z\z\z\z}{\X}
\end{tabulaturo}
 % ⚙

\renewcommand{\notesname}{}
\hesection{קישורים\hfill{\eosectionfont Ligiloj}}
\LTRmulticolcolumns
\begin{LTR}
	\begin{multicols}{2}
		\theendnotes
	\end{multicols}
\end{LTR}
\clearpage
